\documentclass[aspectratio=169]{beamer}

\usepackage[utf8]{inputenc}
\usepackage{natbib}
\renewcommand{\familydefault}{\sfdefault}

\title{Imposter Syndrome}
\author{Bianca Gibson}
\institute{Pycon AU 2016}
\date{}

\begin{document}

\frame{\titlepage}

\begin{frame}
  \begin{center}
    \Huge "the experience of fraudulent thoughts and feelings and the inability to attribute and internalize personal achievement"
    \cite{hh15}
  \end{center}
\end{frame}

\begin{frame}
  \begin{center}
    \Huge My Story
  \end{center}
\end{frame}

\begin{frame}
  \begin{center}
    \Huge Are any of these true for you?
  \end{center}
\end{frame}

\begin{frame}
  \begin{center}
    \Huge  I can give the impression that I’m more competent than I really am
  \end{center}
\end{frame}

\begin{frame}
  \begin{center}
    \Huge  When  people  praise  me  for  something  I’ve  accomplished,  I’m  afraid  I  won’t  be able to live up to their expectations of me in the future.
  \end{center}
\end{frame}

\begin{frame}
  \begin{center}
    \Huge  I sometimes think I obtained my present position or gained my present success because I happened to be in the right
    place at the right time or knew the right people.
  \end{center}
\end{frame}

\begin{frame}
  \begin{center}
    \Huge  I’m  afraid  people  important  to  me  may  find  out  that  I’m  not  as  capable  as  they  think  I  am.
  \end{center}
\end{frame}

\begin{frame}
  \begin{center}
    \Huge I tend to remember the in
    cidents in which I have not done my best more than those times I have done my best
  \end{center}
\end{frame}

\begin{frame}
  \begin{center}
    \Huge  I rarely do a project or task as well as I’d like to do it.
  \end{center}
\end{frame}

\begin{frame}
  \begin{center}
    \Huge  Sometimes I feel or believe that my success in my life or in my job has been the result of some kind of error.
  \end{center}
\end{frame}

\begin{frame}
  \begin{center}
    \Huge    It’s  hard  for  me  to  accept  compliments  or  praise  about  my  intelligence  or  accomplishments
  \end{center}
\end{frame}

\begin{frame}
  \begin{center}
    \Huge   Sometimes I’m afraid others will discover how much knowledge or ability I really lack
  \end{center}
\end{frame}

\begin{frame}
  \begin{center}
    \Huge Clance Imposter Phenomenon scale
  \end{center}
\end{frame}

\begin{frame}
  \begin{center}
    \Huge Who has it?
% initially discovered in high achieving women studying and in academia
% later found in to be in other populations too
% http://www.paulineroseclance.com/pdf/ip_high_achieving_women.pdf
  \end{center}
\end{frame}

\begin{frame}
  \begin{center}
    \Huge What are the characteristics?
    % Clance (1985) as cited in Sakulku (Jarawan)
  \end{center}
\end{frame}

\begin{frame}
  \begin{center}
    \Huge Imposter cycle (photo)
  \end{center}
\end{frame}

\begin{frame}
  \begin{center}
    \Huge Desire to be the best
    % will be the biggest fish in a small pond (school)
    % then go to uni - lots of bright people, not the best anymore
    % conclude that they are stupid because they aren't the best anymore
  \end{center}
\end{frame}

\begin{frame}
  \begin{center}
    \Huge Superman/Superwoman aspects
    % perfectionism
  \end{center}
\end{frame}

\begin{frame}
  \begin{center}
    \Huge Denial of competence and Discounting praise
  \end{center}
\end{frame}

\begin{frame}
  \begin{center}
    \Huge Fear and guilt about success
  \end{center}
\end{frame}


\begin{frame}
  \begin{center}
    \Huge What childhood circumstances create it?
% Generally either has a sibling or close relative that was the designated 'intelligent' family member
% the woman is then told that she is the 'sensitive' or socially adept one, not the smart one
% if told that because they are a woman then stereotype threat?
% or, told that they are superior in every way and success will come easily
% Then they can't cope with when it doesn't
  \end{center}
\end{frame}

\begin{frame}
  \begin{center}
    \Huge Personality traits contribute?
% "common among individuals with particular personality traits (e.g. neurot-
    icism, conscientiousness, achievement-orientation), have perfectionist expectations over work"
    Outing the imposter: study exploring IP among HE faculty
    DOI:  10.1002/nha3.20098
  \end{center}
\end{frame}

\begin{frame}
  \begin{center}
    \Huge Work circumstances that contribute
% "and who work in highly competitive, stressful occupations similar to that of the academic environment"
% what about peer review?
% minorities in the field?
    Outing the imposter: study exploring IP among HE faculty
    DOI:  10.1002/nha3.20098
  \end{center}
\end{frame}

\begin{frame}
  \begin{center}
    \Huge How does it work?
% attribute successes to outside factors - luck, colleagues
% attribute failures to themselves
% found to be far more common in women than men
% women viewing ourselves as phony is consistent with societal view that we aren't competent
% easier to not internalize success than go against the views of society!
% often seem to perservere despite that, some desire to prove society and themselves wrong
  \end{center}
\end{frame}

\begin{frame}
  \begin{center}
    \Huge 4 Behaviours that preserve it
    \begin{itemize}
      \item Diligence
      \item Intellectual Inauthenticity
      \item Use of charm
      \item Avoiding displays of confidence
    \end{itemize}
  \end{center}
\end{frame}

\begin{frame}
  \begin{center}
    \Huge Diligence

    "She develops an unstated but vaguely aware belief that if she were to think she could succeed she would  actually  fail."
% fear getting caught, so work hard to prevent discovery
% when we get success and approval we dismiss it as just because we worked so hard
% feels temporary elation on success
% success on the surface, but underlying phoniness remains
  \end{center}
\end{frame}

\begin{frame}
  \begin{center}
    \Huge Intellectual Inauthenticity
% chose not to reveal ideas or opinions
% tell people what they want to hear
% intellectual flattery - writing according to their teachers' biases
% or for a developer - implementing it how the more senior developers or tech lead would want, not what you think is best
% remaining silent in face of opposing view points
% prevents them from finding out what people would think of their authentic views
% maintains imposter syndrome
  \end{center}
\end{frame}

\begin{frame}
  \begin{center}
    \Huge Charm
    "Typically, she believes, "I am stupid," but at anther
    level  she  believes  she  is  brilliant,  creative,  and  special  if  only  the  right  person  would  discover
    her genius and thereby help her believe in her intellect. "
% aim to be liked as well as recognised intellectually
% finds a candidate she respects, then tries to impress to gain approval
% studies them, figures out how to impress them, sets about winning them over
% "For  example,  if  the  potential
%mentor  grows  mushrooms,  the  woman
%  will  make  it  her  business  to  be  able  to  converse
%  enthusiastically  and  knowledgably  about  mushroom  growing.    "
% listens with understanding and concern
% usually gains approval, but doesn't work
% will never believe the praise because it's based on liking her
% if she was really that bright, would she need the outside approval?
  \end{center}
\end{frame}

\begin{frame}
  \begin{center}
    \Huge Avoiding displays of confidence
% find a recent quote for success going against likeability for women
% Many women have a motive to avoid success, a well justified fear of rejection or less feminine if they do
% look at all the crap Julia Gillard put up with - quote from her speech
% pattern of tight roping
% denying their success allows them to live out achievement orientation while allaying some fears about being a successful women
  \end{center}
\end{frame}

\begin{frame}
  \begin{center}
    \Huge What are the consequences?
% Outing the imposter: study exploring IP among HE faculty
% DOI:  10.1002/nha3.20098
  \end{center}
\end{frame}

\begin{frame}
  \begin{center}
    \Huge Bouts of depression and anxiety
  \end{center}
\end{frame}


\begin{frame}
  \begin{center}
    \Huge Psychological distress
  \end{center}
\end{frame}

\begin{frame}
  \begin{center}
    \Huge Low self confidence
  \end{center}
\end{frame}

\begin{frame}
  \begin{center}
    \Huge adversely related to job well-being, satisfaction and performance
  \end{center}
\end{frame}

\begin{frame}
  \begin{center}
    \Huge Ways to cope
  \end{center}
\end{frame}

\begin{frame}
  \begin{center}
    \Huge Mentoring
% They can normalise the feelings
% emotional support
% instrumental support
% challenge to accept praise
  \end{center}
\end{frame}

\begin{frame}
  \begin{center}
    \Huge
  \end{center}
\end{frame}

\begin{frame}
  \begin{center}
    \Huge
  \end{center}
\end{frame}

\begin{frame}
  \begin{center}
    \Huge
  \end{center}
\end{frame}

\begin{frame}
  \begin{center}
    \Huge
  \end{center}
\end{frame}

\begin{frame}
  \begin{center}
    \Huge
  \end{center}
\end{frame}

\begin{frame}
  \begin{center}
    \Huge
  \end{center}
\end{frame}

\begin{frame}
  \begin{center}
    \Huge
  \end{center}
\end{frame}

\begin{frame}
  \begin{center}
    \Huge
  \end{center}
\end{frame}

\begin{frame}
  \begin{center}
    \Huge
  \end{center}
\end{frame}

\begin{frame}
  \begin{center}
    \Huge
  \end{center}
\end{frame}

\begin{frame}
\bibliographystyle{apacite}

\bibliography{./slides}
\end{frame}

\end{document}
