\documentclass[aspectratio=169]{beamer}

\usepackage[utf8]{inputenc}
\usepackage[natbibapa]{apacite}
\usepackage{fancyhdr}
\renewcommand{\familydefault}{\sfdefault}

\title{Imposter Syndrome}
\author{Bianca Gibson}
\institute{Pycon AU 2016}
\date{}

\begin{document}

\frame{\titlepage}

\begin{frame}
  \begin{center}
    \Huge `the experience of fraudulent thoughts and feelings and the inability to attribute and internalize personal achievement'
    \\ \small \cite{hh15}
  \end{center}
\end{frame}

\begin{frame}
  \begin{center}
    \includegraphics[scale=.5]{./assets/dog.jpg}
  \end{center}
\end{frame}

\begin{frame}
  \begin{center}
    \Huge Do you have it?
  \end{center}
\end{frame}

\begin{frame}
  \begin{center}
    \Huge How does it work?
  \end{center}
\end{frame}

\begin{frame}
  \begin{center}
    \Huge Consequences
  \end{center}
\end{frame}

\begin{frame}
  \begin{center}
    \Huge Dealing with it
  \end{center}
\end{frame}

\begin{frame}
  \begin{center}
    \Huge Are these true for you?
    \\ \small \cite{clance85}
  \end{center}
\end{frame}

\begin{frame}
  \begin{center}
    \Huge  I have often succeeded on a test or task even though I was afraid that I would not do well before I undertook the task
  \end{center}
\end{frame}

\begin{frame}
  \begin{center}
    \Huge  I can give the impression that I’m more competent than I really am
  \end{center}
\end{frame}

\begin{frame}
  \begin{center}
    \Huge  I avoid evaluations if possible and have a dread of others evaluating me
  \end{center}
\end{frame}

\begin{frame}
  \begin{center}
    \Huge  When  people  praise  me  for  something  I’ve  accomplished,  I’m  afraid  I  won’t  be able to live up to their expectations of me in the future
  \end{center}
\end{frame}

\begin{frame}
  \begin{center}
    \Huge  I sometimes think I obtained my present position or gained my present success because I happened to be in the right place at the right time or knew the right people
  \end{center}
\end{frame}

\begin{frame}
  \begin{center}
    \Huge  I’m  afraid  people  important  to  me  may  find  out  that  I’m  not  as  capable  as  they  think  I  am
  \end{center}
\end{frame}

\begin{frame}
  \begin{center}
    \Huge I tend to remember the incidents in which I have not done my best more than those times I have done my best
  \end{center}
\end{frame}

\begin{frame}
  \begin{center}
    \Huge  I rarely do a project or task as well as I’d like to do it
  \end{center}
\end{frame}

\begin{frame}
  \begin{center}
    \Huge  Sometimes I feel or believe that my success in my life or in my job has been the result of some kind of error
  \end{center}
\end{frame}

\begin{frame}
  \begin{center}
    \Huge    It’s  hard  for  me  to  accept  compliments  or  praise  about  my  intelligence  or  accomplishments
  \end{center}
\end{frame}

\begin{frame}
  \begin{center}
    \Huge At times, I feel my success has been due to some kind of luck
  \end{center}
\end{frame}

\begin{frame}
  \begin{center}
    \Huge   I’m  disappointed  at  times  in  my  present  accomplishments  and  think  I should have accomplished much more
\end{center}
\end{frame}

\begin{frame}
  \begin{center}
    \Huge   Sometimes I’m afraid others will discover how much knowledge or ability I really lack
  \end{center}
\end{frame}

\begin{frame}
  \begin{center}
    \Huge     I’m  often  afraid  that  I  may  fail  at  a  new  assignment  or  undertaking  even  though  I  generally  do  well  at  what  I
  attempt
\end{center}
\end{frame}

\begin{frame}
  \begin{center}
    \Huge   When  I’ve  succeeded  at  something  and  received  recognition  for  my  accomplishments,  I  have  doubts  that I can keep repeating that success
\end{center}
\end{frame}

\begin{frame}
  \begin{center}
    \Huge     If  I  receive  a  great  deal  of  praise  and  recognition  for  something  I’ve  accomplished,  I  tend  to  discount  the  importance
  of  what  I’ve  done
\end{center}
\end{frame}

\begin{frame}
  \begin{center}
    \Huge     I often compare my ability to those around me and think they may be more intelligent than I am
  \end{center}
\end{frame}

\begin{frame}
  \begin{center}
    \Huge      I often worry about not succeeding with a project or examination, even though others around me have considerable
confidence that I will do well
\end{center}
\end{frame}

\begin{frame}
  \begin{center}
    \Huge       If  I’m  going  to  receive  a  promotion  or  gain  recognition  of  some  kind,  I  hesitate  to  tell  others  until  it  is  an
  accomplished fact
\end{center}
\end{frame}

\begin{frame}
  \begin{center}
    \Huge  I  feel  bad  and  discouraged  if  I’m  not  `the  best'  or  at  least  `very  special'  in  situations  that  involve  achievement
\end{center}
\end{frame}

\begin{frame}
  \begin{center}
    \Huge Who has it?
    \\ \small \cite{clanceimes78}
    \\ \small \cite{attr98}
    \\ \small \cite{colour}
%     initially discovered in high achieving women studying and in academia
%     later found in to be in other populations too
%     Maybe higher in women, research goes both ways
%     Maybe higher in some ethnic groups, studied in African Americans
  \end{center}
\end{frame}

\begin{frame}
  \begin{center}
    \Huge Minorities
    \\ \small \cite{apa13}
  \end{center}
\end{frame}

\begin{frame}
  \begin{center}
    \Huge New endeavour
    \\ \small \cite{apa13}
     neighbourhood organiser and suggesting I be a council candidate
  \end{center}
\end{frame}

\begin{frame}
  \begin{center}
    \Huge What are the characteristics?
  \end{center}
\end{frame}

\begin{frame}
  \begin{center}
    \includegraphics[scale=.5]{./assets/clance-impostor-cycle.png}
    \\ \small \cite{sakulku11}
  \end{center}
\end{frame}

\begin{frame}
  \begin{center}
    \Huge How does it work?
    \small
%     attribute successes to outside factors - luck, colleagues \\
%     attribute failures to themselves \\
%     found to be far more common in women than men \\
%     women viewing ourselves as phony is consistent with societal view that we aren't competent \\
%     worse for African American women \\
%     easier to not internalize success than go against the views of society! \\
%     often believe that intelligence is fixed rather than malleable \\
%     motivated by performance goals, try to prove intelligence \\
%     when fail - react 'helpless' way, blame selves, withdraw from task, anxiety, shame \\
%     overriding concern with others' impressions, idealised self image \\
%     self worth unusually dependent on others - external validation goes away, fall apart \\
    \\ \small \cite{hh15}
    \\ \small \cite{langford93}
    \\ \small \cite{colour}
  \end{center}
\end{frame}

\begin{frame}
  \begin{center}
    \Huge Success does not fix it
    \\ \small \cite{clanceimes78}
    \\ \small \cite{sakulku11}
     % because they dismiss the success
     % disregard if there is any gap between their expectations and performance
     % repetitions of success show dif between actual and ideal standards, make it worse
     % deny our competence, discount praise
  \end{center}
\end{frame}

\begin{frame}
  \begin{center}
    \Huge Desire to be the best
    \small
     % will be the biggest fish in a small pond (school)
     % then go to uni - lots of bright people, not the best anymore
     % conclude that they are stupid because they aren't the best anymore
    \\ \small \cite{sakulku11}
  \end{center}
\end{frame}

\begin{frame}
  \begin{center}
    \Huge Fear and guilt about success
    \\ \small \cite{sakulku11}
  \end{center}
\end{frame}

\begin{frame}
  \begin{center}
    \Huge Defendence
    \\ \small \cite{langford93}
     % mistrusting others
  \end{center}
\end{frame}

\begin{frame}
  \begin{center}
    \Huge Low affiliation
    \\ \small \cite{langford93}
     % in women. enjoyable involvement with other people
  \end{center}
\end{frame}

\begin{frame}
  \begin{center}
    \Huge Low play
    \\ \small \cite{langford93}
     % don't do things for fun
  \end{center}
\end{frame}

\begin{frame}
  \begin{center}
    \Huge Impulsivity
    \\ \small \cite{langford93}
     % low in women, high in men
  \end{center}
\end{frame}

\begin{frame}
  \begin{center}
    \Huge Need for change
    \\ \small \cite{langford93}
     % low in women, high in men
  \end{center}
\end{frame}

\begin{frame}
  \begin{center}
    \Huge Low need for order
    \\ \small \cite{langford93}
     % in men
  \end{center}
\end{frame}

\begin{frame}
  \begin{center}
    \Huge What childhood circumstances create it?
    \\ \small \cite{sakulku11}
    \\ \small \cite{langford93}
     % Generally either has a sibling or close relative that was the designated 'intelligent' family member \\
     % the woman is then told that she is the 'sensitive' or socially adept one, not the smart one \\
     % or, told that they are superior in every way and success will come easily \\
     % Then they can't cope with when it doesn't \\
     % Family valueing success with little effort \\
     % descrepency between feedback and actual success \\
     % lack of positive reinforcement - ``nothing you do is ever good enough'' \\
     % IP higher when family cohesion and expressiveness are low, family conflict and control high. Accounted for 12 of variation \\
     % if not supported or approved may feel achievements are dismissed, unimpressive, unimportant. \\
     % shame, humiliation and inauthenticity common with lack of +ve reinforcement \\
     % IP highly correlated with need to please others in family \\
     % try to live up to idealised image to win approval \\
  \end{center}
\end{frame}

\begin{frame}
  \begin{center}
    \Huge Personality traits
    \small
     % "common among individuals with particular personality traits (e.g. neuroticism, achievement-orientation), have perfectionist expectations over work"
     % inverse with conscientiousness
    \\ \small \cite{hh15}
    \\ \small \cite{sakulku11}
  \end{center}
\end{frame}

\begin{frame}
  \begin{center}
    \Huge Work circumstances that contribute
    \small
     % highly competitive, stressful occupations \\
     % what about peer review? \\
     % higher in untenured faculty - probably maps to staff on fixed term contracts \\
     % not studied in tech, but higher in systems librarians than other librarians \\
     % high tech knowledge requirements, constant technical change, feel out of date \\
     % translate to devs - expectation to keep up with emerging tech \\
     % tendency to focus on what colleagues know that we don't \\
     % how many new js frameworks should we learn per week? \\
    \\ \small \cite{hh15}
    \\ \small \cite{clark14}
  \end{center}
\end{frame}

\begin{frame}
  \begin{center}
    \Huge Racial issues
    \\ \small \cite{colour}
     % studied in African Americans \\
     % people's presumed incompetence in African American women \\
     % vital for them to establish self worth and self reliance - others assessment will be unfairly negative \\
     % Group counselling with other African American women is very effective \\
     % more comfortable with people like them, see the ridicilousness of others IP \\
     % similar situation - share strategies \\
  \end{center}
\end{frame}

\begin{frame}
  \begin{center}
    \Huge Self presentation
    \\ \small \cite{sakulku11}
     % Do not want to appear imperfect, but actually openly disclose their imperfection.\\
     % Is it an interpersonal strategy rather than self evalution?\\
     % could be to avoid negative interpersonal implications of future failures\\
     % only express lower performance expectations when they know others see it\\
     % correlated with other favourale impression management strategies\\
     % makes *lots* of sense for women in tech, since being seen as competetent makes you less likeable\\
  \end{center}
\end{frame}

\begin{frame}
  \begin{center}
    \Huge Behaviours that preserve it
  \end{center}
\end{frame}

\begin{frame}
  \begin{center}
    \Huge Intellectual Inauthenticity
    \\ \small \cite{clanceimes78}
     % chose not to reveal ideas or opinions\\
     % tell people what they want to hear\\
     % intellectual flattery - writing according to their teachers' biases\\
     % or for a developer - implementing it how the more senior developers or tech lead would want, not what you think is best\\
     % remaining silent in face of opposing view points\\
     % prevents them from finding out what people would think of their authentic views\\
     % maintains imposter syndrome\\
  \end{center}
\end{frame}

\begin{frame}
  \begin{center}
    \Huge Charm
    \\ \small \cite{clanceimes78}
     % aim to be liked as well as recognised intellectually\\
     % finds a candidate she respects, then tries to impress to gain approval\\
     % studies them, figures out how to impress them, sets about winning them over\\
     % may pick up ttheir hobbies listens with understanding and concern\\
     % usually gains approval, but doesn't work\\
     % will never believe the praise because it's based on liking her\\
     % if she was really that bright, would she need the outside approval?\\
  \end{center}
\end{frame}

\begin{frame}
  \begin{center}
    \Huge Avoiding displays of confidence
    \\ \small \cite{clanceimes78}
     % Many women have a motive to avoid success, a well justified fear of rejection or being seen as less feminine \\
     % denying their success allows them to live out achievement orientation while allaying some fears about being a successful women \\
  \end{center}
\end{frame}

\begin{frame}
  \begin{center}
    \Huge What are the consequences?
  \end{center}
\end{frame}

\begin{frame}
  \begin{center}
    \Huge Poor mental health
    \\ \small \cite{sakulku11}
  \end{center}
\end{frame}

\begin{frame}
  \begin{center}
    \Huge Bouts of depression and anxiety
    \\ \small \cite{hh15}
  \end{center}
\end{frame}

\begin{frame}
  \begin{center}
    \Huge Emotional exhaustion
    \\ \small \cite{hh15}
     % stress part of burnout - fatigue, depression, emotional and cognitive distancing - low work satisfaction and performance
  % \end{center}
\end{frame}


\begin{frame}
  \begin{center}
    \Huge Psychological distress
    \\ \small \cite{hh15}
  \end{center}
\end{frame}

\begin{frame}
  \begin{center}
    \Huge Low self confidence
    \\ \small \cite{hh15}
  \end{center}
\end{frame}

\begin{frame}
  \begin{center}
    \Huge Lower job well-being, satisfaction and performance
    \\ \small \cite{hh15}
  \end{center}
\end{frame}

\begin{frame}
  \begin{center}
    \Huge Low self-efficacy
    \\ \small \cite{feedback}
     % attribute initial success to ability - higher self efficacy \\
     % higher self efficacy related to higher performance \\
     % performance analysis in early life super important - provides anchor, influencing later attribution \\
  \end{center}
\end{frame}

\begin{frame}
  \begin{center}
    \Huge Ways to cope
  \end{center}
\end{frame}

\begin{frame}
  \begin{center}
    \Huge Mentoring
    \\ \small \cite{hh15}
     % They can normalise the feelings \\
     % emotional support \\
     % instrumental support (tangible help with specific problems) \\
     % challenge to accept praise \\
     % even helpful if they don't discuss IP \\
     % mentors can be the target of unfair comparisons - 67\% don't directly discuss with mentor \\
  \end{center}
\end{frame}

\begin{frame}
  \begin{center}
    \Huge Positive reinforcement
    \\ \small \cite{hh15}
  \end{center}
\end{frame}

\begin{frame}
  \begin{center}
    \includegraphics[scale=.07]{./assets/IMG_20160809_162050.jpg}
    % covering different contributions from organising to office quiz contributions
  \end{center}
\end{frame}

\begin{frame}
  \begin{center}
    \Huge Identify feelings
    \\ \small \cite{caltech}
     % awareness is first step to change
  \end{center}
\end{frame}

\begin{frame}
  \begin{center}
    \Huge Reality check
    \\ \small \cite{caltech}
     % question the thoughts
  \end{center}
\end{frame}

\begin{frame}
  \begin{center}
    \Huge Differentiate between feelings and reality
    \\ \small \cite{caltech}
  \end{center}
\end{frame}

\begin{frame}
  \begin{center}
    \Huge Humor
    \\ \small \cite{hh15}
  \end{center}
\end{frame}

\begin{frame}
  \begin{center}
    \Huge Distracting thoughts or activities
    \\ \small \cite{hh15}
  \end{center}
\end{frame}

\begin{frame}
  \begin{center}
    \Huge Social support
    \\ \small \cite{caltech}
     discuss feelings, get perspective
  \end{center}
\end{frame}

\begin{frame}
  \begin{center}
    \Huge Reduce dependency
    \\ \small \cite{langford93}
     % on external validation for self esteem
     % internalise self worth
  \end{center}
\end{frame}

\begin{frame}
  \begin{center}
    \Huge What to do?
    \small
  \end{center}
\end{frame}

\begin{frame}
  \begin{center}
    \Huge Talk about it
    \small
     % discuss it in your workplace, like the group therapy sessions
  \end{center}
\end{frame}

\begin{frame}
  \begin{center}
    \Huge Be aware when people show it
    \small
     % yourself and others \\
     % I had such bad IP I thought 'they have IP, but I'm right!'
  \end{center}
\end{frame}

\begin{frame}
  \begin{center}
    \Huge Challenge people to accept positive feedback
  \end{center}
\end{frame}

\begin{frame}
  \begin{center}
    \Huge Get perspective
    \small
     % from people that will evaluate you fairly \\
     % don't charm them
  \end{center}
\end{frame}

\begin{frame}
  \begin{center}
    \Huge Questions?
    github.com/biancag/imposter-syndrome
    bianca.rachel.gibson@gmail.com
  \end{center}
\end{frame}

\begin{frame}
 \bibliographystyle{apacite}

  \bibliography{./slides}
\end{frame}

\end{document}
